\documentclass{article}

\usepackage{fullpage}
%\usepackage{hyperref}
\usepackage{graphicx}
\usepackage{longtable}
\usepackage{comment}
\usepackage{amsmath}
\usepackage{mdwlist}
%\usepackage{txfonts}
\usepackage{xspace}
\usepackage{amstext}
\usepackage{amssymb}
\usepackage{amsthm}
\usepackage{stmaryrd}
\usepackage{proof}
\usepackage{multicol}
\usepackage{etex}
\usepackage[all, cmtip]{xy}
\usepackage{xcolor}
\usepackage{mathpartir}

% %%%%%%%%%%%%%%%%%%%%%%%%%%%%%%%%%%%%%%%%%%%%%%%%%%%%%%%%%%%%%%%%%%%%%%%%%%%%%
\newtheorem{theorem}{Theorem}
\newtheorem{lemma}{Lemma}
%\newtheorem{definition}[theorem]{Definition}
%\newtheorem{proposition}[theorem]{Proposition}
% \newtheorem{corollary}[theorem]{Corollary}

% \newenvironment{proof}[1][Proof]{\begin{trivlist}
% \item[\hskip \labelsep {\bfseries #1}]}{\end{trivlist}}

%% \newenvironment{definition}[1][Definition]{\begin{trivlist}
%% \item[\hskip \labelsep {\bfseries #1}]}{\end{trivlist}}
%% \newenvironment{example}[1][Example]{\begin{trivlist}
%% \item[\hskip \labelsep {\bfseries #1}]}{\end{trivlist}}
%% \newenvironment{remark}[1][Remark]{\begin{trivlist}
%% \item[\hskip \labelsep {\bfseries #1}]}{\end{trivlist}}

% \newcommand{\qed}{\nobreak \ifvmode \relax \else
%       \ifdim\lastskip<1.5em \hskip-\lastskip
%       \hskip1.5em plus0em minus0.5em \fi \nobreak
%       \vrule height0.75em width0.5em depth0.25em\fi}

% \newcommand{\qed}{\hfill \ensuremath{\Box}}

\newcommand{\eqn}{=_n}
\newcommand{\eqr}{=_r}
\newcommand{\eqv}{=_v}
\newcommand{\lcal}{$\lambda$-calculus\xspace}

%%%%%%%%%%%%%%%%%%%%%%%%%%%%%%%%%%%%%%%%%%%%%%%%%%%%%%%%%%%%%%%%%%%%%%%%%%%%%


\title{B522 : Project - Milestone 4 (Progress)}

\author{Eric Holk (eholk@cs.indiana.edu) \and Saliya Ekanayake (sekanaya@cs.indiana.edu) \and Hyungro Lee (lee212@indiana.edu)}

\begin{document}
\maketitle

\newcommand{\cons}{\ensuremath{\mathtt{cons}}}
\newcommand{\car}{\ensuremath{\mathtt{car}}}
\newcommand{\cdr}{\ensuremath{\mathtt{cdr}}}
\newcommand{\vnull}{\ensuremath{\mathtt{null}}}
\newcommand{\nil}{\ensuremath{\mathtt{nil}}}
\newcommand{\nils}{\ensuremath{\mathit{nil}}}
\newcommand{\addone}{\ensuremath{\mathtt{add1}}}
\newcommand{\gap}{\ensuremath{\quad}}

%%%%%%%%%%%%%%%%%%%%%%%%%%%%%%%%%%%%%%%%%%%%%%%%%%%%%%%%%%%%%%%%%%%%%%%%%%%%%%%%
\section{Syntax for Categorical Abstract Machine (CAM)}

\begin{itemize}

\item Source Syntax

\[
\begin{array}{rlr}
  t & ::= n ~| ~\lambda t ~| ~t_0 t_1 ~| ~nil ~| ~(cons ~t_1 ~t_2) ~| ~(car ~t)
  ~| ~(cdr ~t) & \textrm{(Terms)}
\end{array}
\]


\item Machine Syntax
\[
\begin{array}{rlr}
v & ::= null ~| ~(v_1, ~v_2) ~| ~[v, t] & \textrm{(Values)} \\
k & ::= kempty ~| ~kapp(t, ~k) ~| ~kapp.inner(k) ~| ~kcons(t, ~k) | ~kcons.inner(k) \\
& \gap ~~| ~kcar(k) ~| ~kcdr(k) & \textrm{(Continuations)} \\
\end{array}
\]

\end{itemize}

\section{Operational Semantics}

The semantics are defined by two mutually recursive transition
functions, given below.

\[
\begin{array}{rll}
  t 
  & \Rightarrow_\mathit{eval} & 
  \langle t, \vnull, \nils, \mathit{kempty} \rangle \\
  \hline

  \langle n, v, s, k \rangle 
  & \Rightarrow_\mathit{eval} & 
  \langle k, \gamma(n, v), s \rangle \\

  \langle \lambda t, v, s, k \rangle 
  & \Rightarrow_\mathit{eval} & 
  \langle k, [v, t], s \rangle \\

  \langle \nil, v, s, k \rangle 
  & \Rightarrow_\mathit{eval} & 
  \langle k, \vnull, s \rangle \\

  \langle t_0~t_1, v, s, k \rangle 
  & \Rightarrow_\mathit{eval} & 
  \langle t_0, v, v :: s, \mathit{kapp}(t_1, k) \rangle \\

  \langle (\cons~t_1~t_2), v, s, k \rangle 
  & \Rightarrow_\mathit{eval} & 
  \langle t_1, v, v :: s, \mathit{kcons}(t_2, k) \rangle \\

  \langle (\car~t), v, s, k \rangle 
  & \Rightarrow_\mathit{eval} & 
  \langle t, v, s, \mathit{kcar}(k) \rangle \\

  \langle (\cdr~t), v, s, k \rangle 
  & \Rightarrow_\mathit{eval} & 
  \langle t, v, s, \mathit{kcdr}(k) \rangle \\

  \hline

  \langle \mathit{kapp}(t, k), v, v' :: s \rangle 
  & \Rightarrow_\mathit{cont} & 
  \langle t, v', v :: s, \mathit{kapp.inner}(k) \rangle \\

  \langle \mathit{kapp.inner}(k), v', [v, t] :: s \rangle 
  & \Rightarrow_\mathit{cont} & 
  \langle t, (v, v'), s, k \rangle \\

  \langle \mathit{kcons}(t_1, k), v, v' :: s \rangle 
  & \Rightarrow_\mathit{cont} & 
  \langle t_1, v', v :: s, \mathit{kcons.inner}(k) \rangle \\

  \langle \mathit{kcons.inner}(k), v, v' :: s \rangle 
  & \Rightarrow_\mathit{cont} & 
  \langle k, (v', v), s \rangle \\

  \langle \mathit{kcar}(k), (v_1, v_2), s \rangle 
  & \Rightarrow_\mathit{cont} & 
  \langle k, v_1, s \rangle \\

  \langle \mathit{kcdr}(k), (v_1, v_2), s \rangle 
  & \Rightarrow_\mathit{cont} & 
  \langle k, v_2, s \rangle \\

  \hline

  \langle \mathit{kempty}, v, \nils \rangle 
  & \Rightarrow_\mathit{cont} & 
  v \\

\end{array}
\]

where

\[
\begin{array}{rlr}
\gamma(0, (v_1, v_2)) = v_2 \\
\gamma(n, (v_1, v_2)) = \gamma(n-1, v_1)\\
\end{array}
\]


\section{Type System}

\begin{itemize}

\item Types
\[
\begin{array}{rlr}
  \tau & ::= \tau_\vnull ~|~ \tau \times \tau ~|~ \tau \rightarrow \tau \\  
\end{array}
\]


\item Typing Environments
\[
\begin{array}{rlr}
  \Gamma & ::= \cdot ~|~ \Gamma, \tau \\
\end{array}
\]

Our types consist of the \vnull~ type ($\tau_\vnull$), pair types ($\tau
\times \tau$) and function types ($\tau \rightarrow \tau$). We
considered using the type $\mathrm{list}~\tau$ for the result of the
$\cons$ and $\nil$ terms, but we opted to represent lists as pairs
instead. This gives us the flexibility to store non-uniform data types
in a list. It also allows us to maintain progress for well-typed
terms. For example, we migth say that $(\car~t)$ works on terms of
type $\mathrm{list}~\tau$ and gives you a $\tau$, except that we would
get stuck if we tried to evaluate $(\car~\nil)$, even though this term
would be well-typed.

\item Type environment lookup

\[
\begin{array}{rl}
  (\Gamma, \tau)(0) & = \tau \\
  (\Gamma, \tau)(n + 1) & = \Gamma(n) \\
\end{array}
\]

\item Type judgments for terms

\begin{mathpar}

\inferrule*
{
  \Gamma(n) = \tau
}
{
  \Gamma \vdash n : \tau
}


\inferrule*
{
  \Gamma, \tau_1 \vdash t : \tau_2
}
{
  \Gamma \vdash \lambda t : \tau_1 \rightarrow \tau_2
}

\inferrule*
{
  \Gamma \vdash t_0 : \tau_1 \rightarrow \tau
  \+
  \Gamma \vdash t_1 : \tau_1
}
{
  \Gamma \vdash t_0 ~ t_1 : \tau
}

\inferrule*{
  \Gamma \vdash \nil : \tau_\vnull
}
{}

\inferrule*{
  \Gamma \vdash t_1 : \tau_1
  \+ 
  \Gamma \vdash t_2 : \tau_2 
}
{
  \Gamma \vdash (\cons~t_1~t_2) : \tau_1 \times \tau_2  
}

\inferrule*{
  \Gamma \vdash t : \tau_1 \times \tau_2  
}
{
  \Gamma \vdash (\car~t) : \tau_1  
}

\inferrule*{
  \Gamma \vdash t : \tau_1 \times \tau_2 
}
{
  \Gamma \vdash (\cdr~t) : \tau_2  
}

\end{mathpar}

\item
Type judgments for values

\begin{mathpar}
  \inferrule*{}{\vdash \vnull : \tau_\vnull}

  \infer{\vdash (v_1, v_2) : \tau_1 ~\times ~\tau_2}
  {\vdash v_1:\tau_1 ~~~\vdash v_2:\tau_2}

  \infer{\vdash [v, t]: \tau_1 \rightarrow \tau_2}
  {\Gamma \sim v ~~~ \Gamma, \tau_1 \vdash t:\tau_2}
\end{mathpar}

In order to typecheck values, we appeal to a consistency relation
between typing environments and values. This is because closures are
stored as a value (tuples of values representing the environment) and
a term, and we must typecheck the term in an typing environment that
matches its runtime environment.

\item Type/Value Environment Consistency

  \begin{mathpar}
    \inferrule*
    {}
    {
      \cdot \sim \vnull
    }

    \inferrule*{
      \Gamma \sim v_1
      \+
      \vdash v_2 : \tau
    }
    {
      \Gamma, \tau \sim (v_1, v_2)
    }
        
  \end{mathpar}

  Note that this consistency relationship is mutually recursive with
  the value typing judgment. We must be careful to make sure both are
  well-founded. In both cases, we are recurring on smaller values, so
  the definition is indeed well-founded.

  We have implemented this relation in Coq, and Coq agrees with us
  that the recursion will terminate.

\item Type judgments for continuations

  \begin{mathpar}

\inferrule*{}{nil \vdash kempty : \tau \rightarrow \tau}



\infer{v::s \vdash kapp (t, k) : (\tau_1 \rightarrow \tau_2) \rightarrow \tau}
{\Gamma \sim v ~~~ \Gamma \vdash t:\tau_1 ~~~ s \vdash k : \tau_2 \rightarrow \tau}



\infer{v::s \vdash kapp.inner(k): \tau_1 \rightarrow \tau}
{\vdash v:\tau_1 \rightarrow \tau_2 ~~~ s \vdash k: \tau_2 \rightarrow \tau}



\infer{v::s \vdash kcons (t, k) : \tau_1 \rightarrow \tau}
{\Gamma \sim v ~~~ \Gamma \vdash t:\tau_2 ~~~ s \vdash k:(\tau_1 \times \tau_2) \rightarrow \tau}



\infer{v::s \vdash kcons.inner (k) : \tau_2 \rightarrow \tau}
{\vdash v:\tau_1 ~~~ s \vdash k:(\tau_1 \times \tau_2) \rightarrow \tau}



\infer{s \vdash kcar (k) : (\tau_1 \times \tau_2) \rightarrow \tau}{s \vdash k: \tau_1 \rightarrow \tau}



\infer{s \vdash kcdr (k) : (\tau_1 \times \tau_2) \rightarrow \tau}{s \vdash k: \tau_2 \rightarrow \tau}

\end{mathpar}

\item
Type judgments for machine states

\begin{mathpar}
\infer{\vdash <t,v,s,k>: \tau}
{\Gamma \sim v ~~~ \Gamma \vdash t:\tau_1 ~~~ s \vdash k:\tau_1 \rightarrow \tau}

\infer{\vdash <k,v,s>: \tau}
{s \vdash k:\tau_1 \rightarrow \tau ~~~ \vdash v: \tau_1}


    
  \end{mathpar}

\end{itemize}

\section{Type Safety}

\subsection{Lemmas}

\begin{lemma}[Inversion Lemmas]
  \label{lem:inversion}
  \begin{enumerate} 
  \item If $\Gamma \vdash n : \tau$ then $\Gamma(n) = \tau$.
  \item If $\Gamma \vdash \lambda t : \tau_1 \rightarrow \tau_2$ then 
    $\Gamma, \tau_1 \vdash t : \tau_2$ 
  \item If $\Gamma \vdash t_0 t_1 : \tau$ then $\Gamma \vdash t_0 : \tau_1
    \rightarrow \tau ~and ~\Gamma \vdash t_1 : \tau_1$
  \item If $\Gamma \vdash (cons ~t_1 ~t_2) : \tau_1 \times \tau_2$ then $\Gamma
    \vdash t_1 : \tau_1 ~and ~\Gamma \vdash t_2 : \tau_2$
  \item If $\Gamma \vdash (car ~t) : \tau_1$ then $\Gamma \vdash t: \tau_1
    \times \tau_2$
  \item If $\Gamma \vdash (cdr ~t) : \tau_2$ then $\Gamma \vdash t: \tau_1
   \times \tau_2$
  \item If $(v_1, v_2) :\tau_1 \times \tau_2$ then $\vdash v_1 : \tau_1 ~and
    ~\vdash v_2 : \tau_2$.
  \item If $\vdash [v,t] : \tau_1 \rightarrow \tau_2$ then there
    exists $\Gamma$ such that $\Gamma \sim v$ and $\Gamma, \tau_1
    \vdash t : \tau_2$
  \item If $\Gamma, \tau \sim (v_1, v_2)$ then $\Gamma \sim v_1$ and $\vdash v_2
    : \tau$
  \item If $v::s \vdash kapp (t, k) : (\tau_1 \rightarrow \tau_2) \rightarrow
    \tau$ then there exists $\Gamma$ such that $\Gamma \sim v$ and $\Gamma
    \vdash t:\tau_1$ and $s \vdash k : \tau_2 \rightarrow \tau$
  \item If $v::s \vdash kapp.inner(k): \tau_1 \rightarrow \tau$ then there
   exists $\tau_2$ such that $\vdash v:\tau_1 \rightarrow \tau_2$ and $s \vdash
   k: \tau_2 \rightarrow \tau$
  \item If $v::s \vdash kcons (t, k) : \tau_1 \rightarrow \tau$ then there
   exists $\Gamma, \tau_2$ such that $\Gamma \sim v$ and $\Gamma \vdash
   t:\tau_2$ and $s \vdash k:(\tau_1 \times \tau_2) \rightarrow \tau$
  \item If $v::s \vdash kcons.inner (k) : \tau_2 \rightarrow \tau$ then there exists
   $\tau_1$ such that $\vdash v:\tau_1$ and $s \vdash k:(\tau_1 \times \tau_2)
   \rightarrow \tau$
  \item If  $s \vdash kcar (k) : (\tau_1 \times \tau_2) \rightarrow \tau$ then
   $s \vdash k: \tau_1 \rightarrow \tau$
  \item If $s \vdash kcdr (k) : (\tau_1 \times \tau_2) \rightarrow \tau$ then
   $s \vdash k: \tau_2 \rightarrow \tau$
  \item If $\vdash <t,v,s,k>: \tau$ then there exists $\Gamma, \tau_1$ such that
  $\Gamma \sim v$ and $\Gamma \vdash t:\tau_1$ and $s \vdash k:\tau_1 \rightarrow
   \tau$
  \item If $\vdash <k,v,s>: \tau$ then there exists $\tau_1$ such that $s \vdash
  k:\tau_1 \rightarrow \tau$ and $\vdash v: \tau_1$   
  \end{enumerate}
\end{lemma}
\begin{proof}
  These results follow directly from the definition of the typing
  relations.
\end{proof}


\begin{lemma}
  \label{lem:lookup-consistency}
  If $\Gamma \sim v$ and $\Gamma \vdash n : \tau$ then $\vdash
  \gamma(n, v) : \tau$.
\end{lemma}
\begin{proof}
  By induction on $n$.

  \paragraph{Case $n = 0$} Because $\Gamma(0) = \tau$, we know that
  $\Gamma$ must be of the form $\Gamma', \tau$ for some $\Gamma'$ and
  $\tau$. We also know from the consistency relation that $v$ must be
  of the form $(v_1, v_2)$ for some $v_1, v_2$ such that $\Gamma' \sim
  v_1$ and $\vdash v_2 : \tau$. Since $\gamma(0, (v_1, v_2)) = v_2$,
  we have $\vdash \gamma(0, v) : \tau$.

  \paragraph{Case $n = m + 1$} The induction hypothesis is that for
  any $\Gamma', v'$, and $\tau'$ such that $\Gamma' \sim v'$, if
  $\Gamma' \vdash m : \tau'$ then $\vdash \gamma(m, v') : \tau'$. We
  also assume that we have $\Gamma \sim v$ and $\Gamma \vdash m + 1 :
  \tau$. We need to show $\vdash \gamma(m + 1, v) : \tau$. We know
  that $\Gamma(m + 1) = \tau$, and therefore for some $\Gamma'$ and
  $\tau'$ such that $\Gamma = \Gamma', \tau'$ we have $\Gamma'(m) =
  \tau$ and thus $\Gamma' \vdash m : \tau$. Furthermore, since $\Gamma
  \sim v$, $v$ must have the form $(v_1, v_2)$ with $\Gamma' \sim
  v_1$.

  We needed to show $\vdash \gamma(m + 1, v) : \tau$. If we substitute
  $(v_1, v_2)$ for $v$ and simplify, we now need to show $\vdash
  \gamma(m, v_1) : \tau$. Since we have $\Gamma' \sim v_1$ and
  $\Gamma' m : \tau$, we can invoke the induction hypothesis and
  conclude $\vdash \gamma(m, v_1) : \tau$.

\end{proof}

\begin{lemma}[Canonical Forms]
  We can infer the shape of a value from its type.
  \begin{enumerate}
  \item If $\vdash v : \tau_\vnull$ then $v = \vnull$.
  \item If $\vdash v : \tau_1 \times \tau_2$ then there exists $v_1$,
    $v_2$ such that $v = (v_1, v_2)$.
  \item If $\vdash v : \tau_1 \rightarrow \tau_2$ then there exists
    $v'$, $t$ such that $v = [v, t]$.
  \end{enumerate}
\end{lemma}

\begin{proof}
  This follows directly from the value typing relation.
\end{proof}

\subsection{Preservation}

\begin{theorem}[Preservation]
  \begin{itemize} \hfill
  \item If $\vdash t : \tau$ then $\langle t, null, \nil, kempty \rangle : \tau$ 
  \item If $\vdash \langle t, v, s, k \rangle : \tau$ and $\vdash
    \langle t, v, s, k \rangle : \tau \Rightarrow_\mathit{eval}
    \langle t', v', s', k' \rangle$ then $\vdash \langle t', v', s',
    k' \rangle : \tau$
  \item If $\vdash \langle t, v, s, k \rangle : \tau$ and $\vdash \langle t, v,
  s, k \rangle : \tau  \Rightarrow_\mathit{eval} \langle k', v', s' \rangle$
  then $\vdash \langle k', v', s' \rangle : \tau$
  \item If $\vdash \langle k, v, s \rangle : \tau$ and $\vdash \langle k, v,
  s \rangle : \tau  \Rightarrow_\mathit{eval} \langle t, v', s', k' \rangle$
  then $\vdash \langle t, v', s', k' \rangle : \tau$
  \item If $\vdash \langle k, v, s \rangle : \tau$ and $\vdash \langle k, v,
  s \rangle : \tau  \Rightarrow_\mathit{eval} \langle k', v', s' \rangle$
  then $\vdash \langle k', v', s' \rangle : \tau$
  \item If $\langle kempty, v, \nil \rangle : \tau$ and $\langle kempty, v, \nil
  \rangle : \tau \Rightarrow_\mathit{eval} v$ then $\vdash v : \tau$ 
    
  \end{itemize}
\end{theorem}

\begin{proof}
  By induction on the structure of the typing derivation for machine states.

\paragraph{Case $\vdash t : \tau$} Show that
$\vdash \langle t, null, \nil, kempty \rangle : \tau$.

\[
\inferrule*{
  \inferrule*{
  }{
    \cdot \sim null
  }
  \+
  \inferrule*{
    \textrm{By IH}
  }{
    \cdot \vdash t : \tau
  }
  \+
  \inferrule*{
  }{
  	\nil \vdash kempty : \tau \rightarrow \tau
  }
}{
  \vdash \langle t, null, \nil, kempty \rangle : \tau
}
\]

\paragraph{Case $\vdash \langle t_0 t_1, v, s, k \rangle : \tau$} Show that 
$\vdash \langle t_0, v, v::s, kapp(t_1, k) \rangle: \tau$.

From our inversion lemmas, we know there exists $\Gamma$, $\tau_1$ and $\tau_2$
such that $\Gamma \sim v$, $\Gamma \vdash t_0 t_1 : \tau_1$ and $s \vdash k
: \tau_1 \rightarrow \tau$. Now we will attempt to derive $\vdash
\langle t_0, v, v::s, kapp(t_1, k) \rangle : \tau$.

\[
\inferrule*{
	\inferrule*{
 		\textrm{By IH}
	}{
		\Gamma \sim v
	}
	\+
		\inferrule*{
			\textrm{By IH and Lemma \ref{lem:inversion}}
		}{
			\Gamma \vdash t_0 : \tau_2 \rightarrow \tau_1
		}
	\+
		\inferrule*{
			\inferrule*{
 				\textrm{By IH}
			}{
				\Gamma \sim v
			}
			\+
				\inferrule*{
					\textrm{By Lemma \ref{lem:inversion}}
				}{
					\Gamma \vdash t_1 : \tau_2
				}
			\+
				\inferrule*{
					\textrm{By IH}
				}{
					s \vdash k : \tau_1 \rightarrow \tau
				}
		}{
			v::s \vdash kapp( t_1, k) : (\tau_2 \rightarrow \tau_1) \rightarrow \tau
		}
}{
	\vdash \langle t_0, v, v::s, kapp(t_1, k) \rangle : \tau
}
\]

\paragraph{Case $\vdash \langle (cons ~t_1 ~t_2), v, s, k \rangle : \tau$} Show that 
$\vdash \langle t_1, v, v::s, kcons(t_2, k) \rangle: \tau$.

From our inversion lemmas, we know there exists $\Gamma$, $\tau_1$ and $\tau_2$
such that $\Gamma \sim v$, $\Gamma \vdash (cons ~t_1 ~t_2) : \tau_1 \times \tau_2$ and $s \vdash k
: (\tau_1 \times \tau_2) \rightarrow \tau$. Now we will attempt to derive $\vdash
\langle  t_1, v_1, v::s, kcons(t_2, k) \rangle : \tau$.

\[
\inferrule*{
	\inferrule*{
 		\textrm{By IH}
	}{
		\Gamma \sim v
	}
	\+
		\inferrule*{
			\textrm{By IH}
		}{
			\Gamma \vdash t_1 : \tau_1
		}
	\+
		\inferrule*{
			\inferrule*{
 				\textrm{By IH}
			}{
				\Gamma \sim v
			}
			\+
				\inferrule*{
 					\textrm{By IH}
				}{
					\Gamma \vdash t_2 : \tau_2
				}
			\+
				\inferrule*{
 					\textrm{By IH}
				}{
					s \vdash k : (\tau_1 \times \tau_2) \rightarrow \tau
				}
		}{
			v::s \vdash kcons( t_2, k) : \tau_1 \rightarrow \tau
		}
}{
	\vdash \langle t_1, v_1, v::s, kcons(t_2, k) \rangle : \tau
}
\]

\paragraph{Case $\vdash \langle (car ~t), v, s, k \rangle : \tau$} Show that 
$\vdash \langle t, v, s, kcar(k) \rangle: \tau$.

From our inversion lemmas, we know there exists $\Gamma$, $\tau_1$ and $\tau_2$
such that $\Gamma \sim v$, $\Gamma \vdash (car ~t) : \tau_1$ and $s \vdash k
: \tau_1 \rightarrow \tau$. Now we will attempt to derive $\vdash
\langle t, v, s, kcar(k) \rangle : \tau$.

\[
\inferrule*{
	\inferrule*{
 		\textrm{By IH}
	}{
		\Gamma \sim v
	}
	\+
		\inferrule*{
			\textrm{By IH and Lemma \ref{lem:inversion}}
		}{
			\Gamma \vdash t : \tau_1 \times \tau_2
		}
	\+
		\inferrule*{
				\inferrule*{
 					\textrm{By IH}
				}{
					s \vdash k : \tau_1 \rightarrow \tau
				}
		}{
			s \vdash kcar(k) : (\tau_1 \times \tau_2) \rightarrow \tau
		}
}{
	\vdash \langle t, v, s, kcar(k) \rangle : \tau
}
\]

\paragraph{Case $\vdash \langle (cdr ~t), v, s, k \rangle : \tau$} Show that 
$\vdash \langle t, v, s, kcdr(k) \rangle: \tau$.

From our inversion lemmas, we know there exists $\Gamma$, $\tau_1$ and $\tau_2$
such that $\Gamma \sim v$, $\Gamma \vdash (cdr ~t) : \tau_1$ and $s \vdash k
: \tau_1 \rightarrow \tau$. Now we will attempt to derive $\vdash
\langle t, v, s, kcdr(k) \rangle : \tau$.

\[
\inferrule*{
	\inferrule*{
 		\textrm{By IH}
	}{
		\Gamma \sim v
	}
	\+
		\inferrule*{
			\textrm{By IH and Lemma \ref{lem:inversion}}
		}{
			\Gamma \vdash t : \tau_2 \times \tau_1
		}
	\+
		\inferrule*{
				\inferrule*{
			\textrm{By IH}
				}{
					s \vdash k : \tau_1 \rightarrow \tau
				}
		}{
			s \vdash kcdr(k) : (\tau_2 \times \tau_1) \rightarrow \tau
		}
}{
	\vdash \langle t, v, s, kcdr(k) \rangle : \tau
}
\]

\paragraph{Case $\vdash \langle n, v, s, k \rangle : \tau$} Show that
$\vdash \langle k, \gamma(n,v), s \rangle : \tau$.

From our inversion lemmas, we know there exists $\Gamma$ and $\tau_1$
such that $\Gamma \sim v$, $\Gamma \vdash n : \tau_1$ and $s \vdash k
: \tau_1 \rightarrow \tau$. Now we will attempt to derive $\vdash
\langle k, \gamma(n,v), s \rangle : \tau$.

\[
\inferrule*{
  \inferrule*{
    \textrm{By IH}
  }{
    s \vdash k : \tau_1 \rightarrow \tau
  }
  \+
  \inferrule*{
    \textrm{By Lemma \ref{lem:lookup-consistency}}
  }{
    \vdash \gamma(n,v) : \tau_1
  }
}{
  \vdash \langle k, \gamma(n,v), s \rangle : \tau
}
\]

\paragraph{Case $\vdash \langle \lambda t, v, s, k \rangle : \tau$} Show
that $\vdash \langle k, [v,t], s \rangle: \tau$.

From our inversion lemmas, we know there exists  $\Gamma$, $\tau_1$, and
$\tau_2$ such that,

\[
\inferrule*{
	\Gamma \sim v
	\+
	s \vdash k : (\tau_1 \rightarrow \tau_2) \rightarrow \tau
	\+
	\inferrule*{
		\Gamma, \tau_1 \vdash t : \tau_2
	}{
		\Gamma \vdash \lambda t : \tau_1 \rightarrow \tau_2
	}
}{
	\vdash \langle \lambda t, v, s, k \rangle : \tau
}
\]

Now we will attempt to derive $\vdash \langle  k, [v,t], s \rangle : \tau$.

\[
\inferrule*{
	\inferrule*{
		\textrm{By IH}
	}{
		s \vdash k : (\tau_1 \rightarrow \tau_2) \rightarrow \tau
	}
	\+
	\inferrule*{
		\inferrule*{
			\textrm{By IH}
		}{
			\Gamma \sim v			
		}
		\+
		\inferrule* {
			\textrm{By IH and Lemma \ref{lem:inversion}}				
		}{
			\Gamma,\tau_1 \vdash t : \tau_2
		}
	}{
		\vdash [v,t] : \tau_1 \rightarrow \tau_2
	}
}{
	\vdash \langle  k, [v,t], s \rangle : \tau
}
\]

\paragraph{Case $\vdash \langle nil, v, s, k \rangle : \tau$} Show that 
$\vdash \langle k, null, s \rangle: \tau$.

From our inversion lemmas, we know there exists $\Gamma$ and $\tau_1$
such that $\Gamma \sim v$, $\Gamma \vdash nil : \tau_1$ and $s \vdash k
: \tau_1 \rightarrow \tau$. Now we will attempt to derive $\vdash
\langle k, null, s \rangle : \tau$.

\[
\inferrule*{
  \inferrule*{
    \textrm{By IH}
  }{
    s \vdash k : \tau_1 \rightarrow \tau
  }
  \+
  \inferrule*{
    \textrm{By IH}
  }{
    \vdash null : \tau_1
  }
}{
  \vdash \langle k, null, s \rangle : \tau
}
\]

\paragraph{Case $\vdash \langle kapp(t, k), v, v'::s \rangle : \tau$} Show
that $\vdash \langle t, v', v::s, kapp.inner(k) \rangle: \tau$.

From our inversion lemmas, we know there exists  $\Gamma$, $\tau_1$, and
$\tau_2$ such that,

\[
\inferrule*{
	\inferrule*{
		\Gamma \sim v'
		\+
		\Gamma \vdash t : \tau_1
		\+
		s \vdash k : \tau_2 \rightarrow \tau
	}{
		v'::s \vdash kapp (t, k) : (\tau_1 \rightarrow \tau_2) \rightarrow
    	\tau
	}
	\+
	\vdash v : \tau_1 \rightarrow \tau_2
}{
	\vdash \langle kapp(t, k), v, v'::s \rangle : \tau
}
\]

Now we
will attempt to derive $\vdash \langle t, v', v::s, kapp.inner(k) \rangle: \tau$.

\[
\inferrule*{
	\inferrule*{
		\textrm{By IH}
	}{
		\Gamma \sim v'
	}
	\+
	\inferrule*{
		\textrm{By IH}
	}{
		\Gamma \vdash t : \tau_1
	}
	\+
	\inferrule* {
		\inferrule*{
			\textrm{By IH}
		}{
			\vdash v : \tau_1 \rightarrow \tau_2
		}
		\+
		\inferrule*{
			\textrm{By IH}
		}{
			s \vdash k : \tau_2 \rightarrow \tau
		}
	}{
		v::s \vdash kapp.inner(k) : \tau_1 \rightarrow \tau
	}
}{
	\vdash \langle t, v', v::s, kapp.inner(k) \rangle: \tau
}
\]


\paragraph{Case $\vdash \langle kapp.inner(k), v', [v,t]::s \rangle : \tau$} Show
that $\vdash \langle t, (v,v'), s, k \rangle: \tau$.

From our inversion lemmas, we know there exists  $\Gamma$, $\tau_1$, and
$\tau_2$ such that,

\[
\inferrule*{
	\inferrule*{
		\inferrule*{
			\Gamma \sim v
			\+
			\Gamma, \tau_1 \vdash t : \tau_2
		}{
			\vdash [v,t] : \tau_1 \rightarrow \tau_2
		}
		\+
		s \vdash k : \tau_2 \rightarrow \tau
	}{
		[v,t]::s \vdash kapp.inner(k) : \tau_1 \rightarrow \tau
	}
	\+
	\vdash v' : \tau_1
}{
	\vdash \langle kapp.inner(k), v', [v,t]::s \rangle : \tau
}
\]

Now we
will attempt to derive $\vdash \langle t, (v,v'), s, k \rangle: \tau$.

\[
\inferrule*{
	\inferrule*{
		\inferrule*{
			\textrm{By IH}
		}{
			\Gamma \sim v
		}
		\+
		\inferrule*{
			\textrm{By IH}
		}{
			\vdash v' : \tau_1
		}
	}{
		\Gamma' \sim (v,v')
	}
	\+
	\inferrule*{
		\textrm{By IH}
	}{
		\Gamma' \vdash t : \tau_2
	}
	\+
	\inferrule*{
		\textrm{By IH}
	}{
		s \vdash k : \tau_2 \rightarrow \tau
	}
}{
	\vdash \langle t, (v,v'), s, k \rangle: \tau
}
\]

where $\Gamma ' = \Gamma, \tau_1$


\paragraph{Case $\vdash \langle \mathrm{kcons}(t_1, k), v_1, v'::s \rangle : \tau$}
We need to show $\vdash \langle t_1, v', v::s,\mathrm{kcons.inner}(k)\rangle
: \tau$.

\[
\inferrule*{
  \inferrule*{
    \textrm{By Lemma \ref{lem:inversion}}
    % TODO: we should probably reference which part of the lemma we
    % want.
  }{
    \Gamma \sim v'
  }
  \+
  \inferrule*{
    \textrm{By Lemma \ref{lem:inversion}}
  }{
    \Gamma \vdash t_1 : \tau_2
  }
  \+
  \inferrule*{
      \inferrule*{
        \textrm{By Lemma \ref{lem:inversion}}
      }{
        \vdash v : \tau_1
      }
      \+
      \inferrule*{
        \textrm{By Lemma \ref{lem:inversion}}
      }{
        s \vdash k : (\tau_1 \times \tau_2) \rightarrow \tau
      }
  }{
    v :: s \vdash \mathrm{kcons.inner}(k) : \tau_2 \rightarrow \tau
  }
}{
  \vdash \langle t_1, v', v :: s, \mathrm{kcons.inner}(k) \rangle : \tau
}
\]

\paragraph{Case $\vdash \langle \mathrm{kcons.inner}(k), v, v'::s \rangle : \tau$}
We need to show $\vdash \langle k, (v', v), s \rangle : \tau$.

By induction, we have $v' :: s \vdash \mathrm{kcons.inner} : \tau_1
\rightarrow \tau$. We use the inversion lemmas to conclude $\vdash v'
: \tau_2$ and $s \vdash k : (\tau_2 \times \tau_1) \rightarrow
\tau$. We also know by induction that $\vdash v : \tau_1$. Now we will
derive a type for $\langle k, (v', v), s \rangle$.

\[
\inferrule*{
  s \vdash k : (\tau_2 \times \tau_1) \rightarrow \tau
  \+
  \inferrule*{
    \vdash v' : \tau_2
    \+
    \vdash v : \tau_1
  }{
    \vdash (v', v) : (\tau_2 \times \tau_1)
  }
}{
  \vdash \langle k, (v', v), s \rangle : \tau
}
\]

\paragraph{Case $\vdash \langle kcar(k), (v_1, v_2), s \rangle : \tau$} Show that 
$\vdash \langle k, v_1, s \rangle: \tau$.

By induction, we have $ s \vdash kcar(k) : (\tau_1 \times \tau_2) \rightarrow \tau$ and $ \vdash (v_1, v_2) : \tau_1 \times \tau_2$.
From our inversion lemmas, we know there exists $\vdash v_1 : \tau_1$ and $s \vdash k
: \tau_1 \rightarrow \tau$.
Now we will attempt to derive $\vdash
\langle k, v_1, s \rangle : \tau$.

\[
\inferrule*{
		\inferrule*{
			\textrm{By IH and Lemma \ref{lem:inversion}}
		}{
			s \vdash k : \tau_1 \times \tau
		}
	\+
		\inferrule*{
			\textrm{By IH and Lemma \ref{lem:inversion}}
		}{
			\vdash v_1 : \tau_1
		}
}{
	\vdash \langle k, v_1, s \rangle : \tau
}
\]

\paragraph{Case $\vdash \langle kcdr(k), (v_1, v_2), s \rangle : \tau$} Show that 
$\vdash \langle k, v_2, s \rangle: \tau$.

By induction, we have $ s \vdash kcdr(k) : (\tau_1 \times \tau_2) \rightarrow \tau$ and $ \vdash (v_1, v_2) : \tau_1 \times \tau_2$.
sFrom our inversion lemmas, we know there exists  $\vdash v_2 : \tau_2$ and $s \vdash k
: \tau_2 \rightarrow \tau$. Now we will attempt to derive $\vdash
\langle  k, v_2, s \rangle : \tau$.

\[
\inferrule*{
		\inferrule*{
			\textrm{By IH and Lemma \ref{lem:inversion}}
		}{
			s \vdash k : \tau_2 \times \tau
		}
	\+
		\inferrule*{
			\textrm{By IH and Lemma \ref{lem:inversion}}
		}{
			\vdash v_2 : \tau_2
		}
}{
	\vdash \langle k, v_2, s \rangle : \tau
}
\]

\paragraph{Case $\vdash \langle kempty, v, \nil \rangle : \tau$} Show
that $\vdash v: \tau$.

From our inversion lemmas, we can show,

\[
\inferrule*{
	\inferrule*{		
	}{
		\nil \vdash kempty : \tau \rightarrow \tau
	}
	\+
	\vdash v : \tau
}{
	\vdash \langle kempty, v, \nil \rangle : \tau
}
\]

Now we will attempt to derive $\vdash v : \tau$.

\[
\inferrule*{
	\textrm{By IH}
}{
	\vdash v : \tau
}
\]

\end{proof}

\subsection{Progress}

\begin{theorem}[Evaluation Progress]
  If $\vdash \langle t, v, s, k \rangle : \tau$ then either there
  exists $\langle t', v', s', k' \rangle$ such that $\langle t,
  v, s, k \rangle \Rightarrow_\mathit{eval} \langle t', v', s', k'
  \rangle$ or there exists $\langle k', v', s' \rangle$ such that
  $\vdash \langle t, v, s, k \rangle \Rightarrow_\mathit{eval} \langle
  k', v', s' \rangle$.
\end{theorem}

\begin{proof}
  We consider each case for $t$.  For every production $t ::= t'$ in
  the grammar for terms, there is an evaluation rule of the form
  $\langle t', v, s, k \rangle$. Thus, it is clear that any legal
  evaluation configuration can take a step.

  We will give special consideration to the case where $t = n$. Here,
  in order to step to $\langle k, \gamma(n, v), s \rangle$ we must
  show that $\gamma(n, v)$ is defined. Using our inversion lemmas, we
  conclude that there exists some $\Gamma$ and $\tau'$ such that
  $\Gamma \sim v$ and $\Gamma \vdash n : \tau'$. Using Lemma
  \ref{lem:lookup-consistency}, we see it must be the case that
  $\vdash \gamma(n, v) : \tau'$, and thus we conclude that $\gamma(n,
  v)$ is defined.
\end{proof}

\begin{theorem}[Continuation Progress]
  If $\vdash \langle k, v, s \rangle : \tau$ then either $k =
  \mathit{kempty}$ and $s = \nils$ or there exists $\langle t', v',
  s', k' \rangle$ such that $\langle k, v, s \rangle
  \Rightarrow_\mathit{cont} \langle t', v', s', k' \rangle$ or there
  exists $\langle k', v', s' \rangle$ such that $\vdash \langle k, v,
  s \rangle \Rightarrow_\mathit{cont} \langle k', v', s' \rangle$.
\end{theorem}

\begin{proof}
  By cases on $k$.

  \paragraph{Case $k = \mathit{kcar}(k')$} Since $\vdash \langle
  \mathit{kcar}(k'), v, s \rangle : \tau$, we know that for some
  $\tau'$ we have $s \vdash \mathit{kcar}(k') : \tau' \rightarrow
  \tau$ and $\vdash v : \tau'$. From the typing rule for
  $\mathit{kcar}$, we know that $\tau'$ is of the form $\tau_1 \times
  \tau_2$. Thus, by the canonical forms lemma, we have that $v = (v_1,
  v_2)$ for some $v_1$ and $v_2$. Substituting this into our
  configuration gives us $\langle \mathit{kcar}(k'), (v_1, v_2), s
  \rangle$, which we see steps to $\langle k', v_1, s \rangle$.

  \paragraph{Case $k = \mathit{kcdr}(k')$} Again, since $\vdash
  \langle \mathit{kcdr}(k'), v, s \rangle : \tau$, we know that for
  some $\tau'$ we have $s \vdash \mathit{kcdr}(k') : \tau' \rightarrow
  \tau$ and $\vdash v : \tau'$. From the typing rule for
  $\mathit{kcdr}$, we know that $\tau'$ is of the form $\tau_1 \times
  \tau_2$. Thus, by the canonical forms lemma, we have that $v = (v_1,
  v_2)$ for some $v_1$ and $v_2$. Substituting this into our
  configuration gives us $\langle \mathit{kcdr}(k'), (v_1, v_2), s
  \rangle$, which we see steps to $\langle k', v_2, s \rangle$.
  
  \paragraph{Case $k = \mathit{kapp}(t,k')$} Given $\vdash \langle
  \mathit{kapp}(t,k'), v, s \rangle : \tau$, we know that for some
  $\tau'$ we have $s \vdash \mathit{kapp}(t,k') : \tau' \rightarrow
  \tau$ and $\vdash v : \tau'$. The typing rule for $\mathit{kapp}$
  suggests that $s$ is of the form $v'::s'$. Substituting this into our
  configuration gives us $\langle \mathit{kapp}(t, k'), v, v'::s'
  \rangle$, which we see steps to $\langle t, v', v::s', kapp.inner(k')
  \rangle$.
  
  \paragraph{Case $k = \mathit{kapp.inner}(k')$} Given $\vdash \langle
  \mathit{kapp.inner}(k'), v, s \rangle : \tau$, we know that for some
  $\tau'$ we have $s \vdash \mathit{kapp.inner}(k') : \tau' \rightarrow
  \tau$ and $\vdash v : \tau'$. The typing rule for $\mathit{kapp.inner}$
  suggests that $s$ is of the form $v'::s'$ and $\vdash v' : \tau' \rightarrow
  \tau''$ for some $\tau''$. Thus, by the canonical forms lemma, we have that 
  $v' = [v'',t]$ for some $v''$ and $t$. Substituting this into our
  configuration gives us $\langle \mathit{kapp.inner}(k'), v, [v'',t]::s'
  \rangle$, which we see steps to $\langle t, (v'',v), s', k' \rangle$.

 \paragraph{Case $k = kcons(t,k')$} Since $\vdash \langle kcons(t,k'), v, s \rangle : \tau$, we know that for some $\tau'$ we have $s \vdash kcons(t, k') : \tau' \rightarrow \tau$. From the typing rule for $kcons$, we know that $s$ is the form of $v'::s'$. With $v'::s'$, our configuration shows $\langle kcons (t, k'), v, v'::s' \rangle$, which we see steps to $\langle t,v',v::s', kcons.inner(k) \rangle$.

 \paragraph{Case $k = kcons.inner(k')$} Since $\vdash \langle kcons.inner(k'), v, s \rangle : \tau$, we know that for some $\tau'$ we have $s \vdash kcons.inner(k') : \tau' \rightarrow \tau$. From the typing rule for $kcons.inner$, we know that $s$ is the form of $v'::s'$. With $v'::s'$, our configuration shows $\langle kcons.inner (k'), v, v'::s' \rangle$, which we see steps to $\langle k', (v',v), s' \rangle$.

 \paragraph{Case $k = kempty$} Since $\vdash \langle kempty, v, s
 \rangle : \tau$, we know that $s$ must be $\nils$, by our inversion
 lemmas. Therefore we are already in the machine's final
 configuration.

\end{proof}

\end{document}
