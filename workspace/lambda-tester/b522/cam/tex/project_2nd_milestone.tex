\documentclass{article}

\usepackage{fullpage}
%\usepackage{hyperref}
\usepackage{graphicx}
\usepackage{longtable}
\usepackage{comment}
\usepackage{amsmath}
\usepackage{mdwlist}
\usepackage{txfonts}
\usepackage{xspace}
\usepackage{amstext}
\usepackage{amssymb}
\usepackage{stmaryrd}
\usepackage{proof}
\usepackage{multicol}
\usepackage{etex}
\usepackage[all, cmtip]{xy}
\usepackage{xcolor}
\usepackage{mathpartir}

% %%%%%%%%%%%%%%%%%%%%%%%%%%%%%%%%%%%%%%%%%%%%%%%%%%%%%%%%%%%%%%%%%%%%%%%%%%%%%
\newtheorem{theorem}{Theorem}[section]
\newtheorem{lemma}[theorem]{Lemma}
\newtheorem{definition}[theorem]{Definition}
\newtheorem{proposition}[theorem]{Proposition}
% \newtheorem{corollary}[theorem]{Corollary}

% \newenvironment{proof}[1][Proof]{\begin{trivlist}
% \item[\hskip \labelsep {\bfseries #1}]}{\end{trivlist}}

%% \newenvironment{definition}[1][Definition]{\begin{trivlist}
%% \item[\hskip \labelsep {\bfseries #1}]}{\end{trivlist}}
%% \newenvironment{example}[1][Example]{\begin{trivlist}
%% \item[\hskip \labelsep {\bfseries #1}]}{\end{trivlist}}
%% \newenvironment{remark}[1][Remark]{\begin{trivlist}
%% \item[\hskip \labelsep {\bfseries #1}]}{\end{trivlist}}

% \newcommand{\qed}{\nobreak \ifvmode \relax \else
%       \ifdim\lastskip<1.5em \hskip-\lastskip
%       \hskip1.5em plus0em minus0.5em \fi \nobreak
%       \vrule height0.75em width0.5em depth0.25em\fi}

% \newcommand{\qed}{\hfill \ensuremath{\Box}}

\newcommand{\eqn}{=_n}
\newcommand{\eqr}{=_r}
\newcommand{\eqv}{=_v}
\newcommand{\lcal}{$\lambda$-calculus\xspace}

%%%%%%%%%%%%%%%%%%%%%%%%%%%%%%%%%%%%%%%%%%%%%%%%%%%%%%%%%%%%%%%%%%%%%%%%%%%%%


\title{B522 : Project - Milestone 2}

\author{Eric Holk (eholk@cs.indiana.edu) \and Saliya Ekanayake (sekanaya@cs.indiana.edu) \and Hyungro Lee (lee212@indiana.edu)}

\begin{document}
\maketitle

\newcommand{\cons}{\ensuremath{\mathtt{cons}}}
\newcommand{\car}{\ensuremath{\mathtt{car}}}
\newcommand{\cdr}{\ensuremath{\mathtt{cdr}}}
\renewcommand{\null}{\ensuremath{\mathtt{null}}}
\newcommand{\nil}{\ensuremath{\mathtt{nil}}}
\newcommand{\nils}{\ensuremath{\mathit{nil}}}
\newcommand{\addone}{\ensuremath{\mathtt{add1}}}

%%%%%%%%%%%%%%%%%%%%%%%%%%%%%%%%%%%%%%%%%%%%%%%%%%%%%%%%%%%%%%%%%%%%%%%%%%%%%%%%
\section{Type system for Categorical Abstract Machine (CAM)}


\begin{itemize}

\item Syntax

\[
\begin{array}{rlr}
  t & ::= n ~| ~\lambda t ~| ~t_0 t_1 ~| ~nil ~| ~(cons ~t_1 ~t_2) ~| ~(car ~t) ~| ~(cdr ~t) & \textrm{(Terms)} \\
  v & ::= null ~| ~(v_1, ~v_2) ~| ~[v, t] & \textrm{(Values)} \\
  \tau & ::= \tau_\null ~|~ \tau \times \tau ~|~ \tau \rightarrow \tau & \textrm{(Types)} \\
  k & ::= CONT0 ~| ~CONT1(t, ~k) ~| ~CONT2(k) ~| ~CONT3(t, ~k) | ~CONT4(k) ~| ~CONT5(k) ~| ~CONT6(k) & \textrm{(Continuations)} \\
  \Gamma & ::= \cdot ~|~ \Gamma, \tau & \textrm{(Typing Environments)} \\
\end{array}
\]

Our types consist of the \null~ type ($\tau_\null$), pair types ($\tau
\times \tau$) and function types ($\tau \rightarrow \tau$). We
considered using the type $\mathrm{list}~\tau$ for the result of the
$\cons$ and $\nil$ terms, but we opted to represent lists as pairs
instead. This gives us the flexibility to store non-uniform data types
in a list. It also allows us to maintain progress for well-typed
terms. For example, we migth say that $(\car~t)$ works on terms of
type $\mathrm{list}~\tau$ and gives you a $\tau$, except that we would
get stuck if we tried to evaluate $(\car~\nil)$, even though this term
would be well-typed.

\item Type environment lookup

\[
\begin{array}{rl}
  (\Gamma, \tau)(0) & = \tau \\
  (\Gamma, \tau)(n + 1) & = \Gamma(n) \\
\end{array}
\]

\item Type judgments for terms

\begin{mathpar}

\inferrule*
{
  \Gamma(n) = \tau
}
{
  \Gamma \vdash n : \tau
}


\inferrule*
{
  \Gamma, \tau_1 \vdash t : \tau_2
}
{
  \Gamma \vdash \lambda t : \tau_1 \rightarrow \tau_2
}

\inferrule*
{
  \Gamma \vdash t_0 : \tau_1 \rightarrow \tau
  \+
  \Gamma \vdash t_1 : \tau_1
}
{
  \Gamma \vdash t_0 ~ t_1 : \tau
}

\inferrule*{
  \Gamma \vdash \nil : \tau_\null
}
{}

\inferrule*{
  \Gamma \vdash t_1 : \tau_1
  \+ 
  \Gamma \vdash t_2 : \tau_2 
}
{
  \Gamma \vdash (\cons~t_1~t_2) : \tau_1 \times \tau_2  
}

\inferrule*{
  \Gamma \vdash t : \tau_1 \times \tau_2  
}
{
  \Gamma \vdash (\car~t) : \tau_1  
}

\inferrule*{
  \Gamma \vdash t : \tau_1 \times \tau_2 
}
{
  \Gamma \vdash (\cdr~t) : \tau_2  
}

\end{mathpar}

\item
Type judgments for values

\begin{mathpar}
  \inferrule*{}{\vdash \null : \tau_\null}

  \infer{\vdash (v_1, v_2) : \tau_1 ~\times ~\tau_2}
  {\vdash v_1:\tau_1 ~~~\vdash v_2:\tau_2}

  \infer{\vdash [v, t]: \tau_1 \rightarrow \tau_2}
  {\Gamma \vdash v ~~~ \Gamma, \tau_1 \vdash t:\tau_2}
\end{mathpar}

In order to typecheck values, we appeal to a consistency relation
between typing environments and values. This is because closures are
stored as a value (tuples of values representing the environment) and
a term, and we must typecheck the term in an typing environment that
matches its runtime environment.

\item Type/Value Environment Consistency

  \begin{mathpar}
    \inferrule*
    {}
    {
      \cdot \vdash \null
    }

    \inferrule*{
      \Gamma \vdash v_2
      \+
      \vdash v_2 : \tau
    }
    {
      \Gamma, \tau \vdash (v_1, v_2)
    }
        
  \end{mathpar}

  Note that this consistency relationship is mutually recursive with
  the value typing judgment. We must be careful to make sure both are
  well-founded. In both cases, we are recurring on smaller values, so
  the definition is indeed well-founded.

  We have implemented this relation in Coq, and Coq agrees with us
  that the recursion will terminate.

\item Type judgments for continuations

  \begin{mathpar}

\inferrule*{}{nil \vdash CONT0 : \tau \rightarrow \tau}



\infer{v::s \vdash CONT1 (t, k) : (\tau_1 \rightarrow \tau_2) \rightarrow \tau}
{\Gamma \vdash v ~~~ \Gamma \vdash t:\tau_1 ~~~ s \vdash k : \tau_2 \rightarrow \tau}



\infer{v::s \vdash CONT2(k): \tau_1 \rightarrow \tau}
{\vdash v:\tau_1 \rightarrow \tau_2 ~~~ s \vdash k: \tau_2 \rightarrow \tau}



\infer{v::s \vdash CONT3 (t, k) : \tau_1 \rightarrow \tau}
{\Gamma \vdash v ~~~ \Gamma \vdash t:\tau_2 ~~~ s \vdash k:(\tau_1 \times \tau_2) \rightarrow \tau}



\infer{v::s \vdash CONT4 (k) : \tau_2 \rightarrow \tau}
{\vdash v:\tau_1 ~~~ s \vdash k:(\tau_1 \times \tau_2) \rightarrow \tau}



\infer{s \vdash CONT5 (k) : (\tau_1 \times \tau_2) \rightarrow \tau}{s \vdash k: \tau_1 \rightarrow \tau}



\infer{s \vdash CONT6 (k) : (\tau_1 \times \tau_2) \rightarrow \tau}{s \vdash k: \tau_2 \rightarrow \tau}

\end{mathpar}

\item
Type judgments for machine states

\begin{mathpar}
\infer{\vdash <t,v,s,k>: \tau}
{\Gamma \vdash v ~~~ \Gamma \vdash t:\tau_1 ~~~ s \vdash k:\tau_1 \rightarrow \tau}

\infer{\vdash <k,v,s>: \tau}
{s \vdash k:\tau_1 \rightarrow \tau ~~~ \vdash v: \tau_1}


    
  \end{mathpar}

%%%%%%%%%%%%%%%%%%%%%%%%%%%%%%%%%%%%%%%%%%%%%%%%%%%%%%%%%%%%%%%%%%%%%%%%%%%%%%
\end{itemize}

\section{Example}

We will use a list encoding of numbers. That is, $0 \equiv
\null$, $1 \equiv (\null, \null)$, etc. Thus,
$\addone$ is defined as follows.

\[
\addone \equiv \lambda (\cons~\nil~0)
\]

\[
(\addone (\addone 2)) \equiv 
((\lambda (\cons~\nil~0))
 ~((\lambda (\cons~\nil~0))~(cons~\nil~(cons~\nil~\nil))))
\]

The evaluation steps are:
\newcommand{\estep}{\ensuremath{\Rightarrow_\mathit{eval}}}
\newcommand{\cont}[1]{\ensuremath{\mathrm{CONT{#1}}}}
\newcommand{\br}{\ensuremath{\quad}}

\[
\begin{array}{crl}
  % TODO: Saliya - Give type derivations for each of these
  % intermmediate states.
  (1) & & ((\lambda (\cons~\nil~0))
  ~((\lambda (\cons~\nil~0))~(\cons~\nil~\cons~\nil~\nil)))) \\

  (2) & \estep & \langle
  ((\lambda (\cons~\nil~0))
  ~((\lambda (\cons~\nil~0))~(\cons~\nil~(\cons~\nil~\nil)))),
  \null,
  \nils,
  \cont{0}
  \rangle \\
  
  (3) & \estep & \langle
  \lambda (\cons~\nil~0),
  \null,
  \null :: \nils,
  \cont{1}((\lambda (\cons~\nil~0))~(\cons~\nil~(\cons~\nil~\nil)),
           \cont{0})
  \rangle \\

  (4) & \estep & \langle 
  \cont{1}((\lambda (\cons~\nil~0))~(\cons~\nil~(\cons~\nil~\nil)),
  \cont{0}),
  [\null, (\cons~\nil~0)],
  \null :: \nils
  \rangle \\

  (5) & \estep & \langle
  (\lambda (\cons~\nil~0))~(\cons~\nil~(\cons~\nil~\nil)),
  \null,
  [\null, (\cons~\nil~0)] :: \nils,
  \cont{2}(\cont{0})
  \rangle \\

  (6) & \estep & \langle
  \lambda (\cons~\nil~0),
  \null,
  \null :: [\null, (\cons~\nil~0)] :: \nils,
  \cont{1}((\cons~\nil~(cons~\nil~\nil)), \cont{2}(\cont{0}))
  \rangle \\

  (7) & \estep & \langle
  \cont{1}((\cons~\nil~(\cons~\nil~\nil)), \cont{2}(\cont{0})),
  [\null, (\cons~\nil~0)],
  \null :: [\null, (\cons~\nil~0)] :: \nils
  \rangle \\

  (8) & \estep & \langle
  (\cons~\nil~(\cons~\nil~\nil)),
  \null,
  [\null, (\cons~\nil~0)] :: [\null, (\cons~\nil~0)] :: \nils,
  \cont{2}(\cont{2}(\cont{0}))
  \rangle \\

  (9) & \estep & \langle
  \nil,
  \null,
  \null :: [\null, (\cons~\nil~0)] :: [\null, (\cons~\nil~0)] :: \nils, \\
  & \br &
  \cont{3}((\cons~\nil~\nil), \cont{2}(\cont{2}(\cont{0})))
  \rangle \\

  (10) & \estep & \langle
  \cont{3}((\cons~\nil~\nil), \cont{2}(\cont{2}(\cont{0}))),
  \null,\\
  & \br &
  \null :: [\null, (\cons~\nil~0)] :: [\null, (\cons~\nil~0)] :: \nils
  \rangle \\

  % TODO: Eric - Do type derivations for the next 10 states.

  (11) & \estep & \langle
  (\cons~\nil~\nil),
  \null,
  \null :: [\null, (\cons~\nil~0)] :: [\null, (\cons~\nil~0)] :: \nils,\\
  & \br &
  \cont{4}(\cont{2}(\cont{2}(\cont{0})))
  \rangle \\

  (12) & \estep & \langle
  \nil,
  \null,
  \null :: \null :: [\null, (\cons~\nil~0)] :: [\null, (\cons~\nil~0)] :: \nils,\\
  & \br &
  \cont{3}(\nil, \cont{4}(\cont{2}(\cont{2}(\cont{0}))))
  \rangle \\

  (13) & \estep & \langle
  \cont{3}(\nil, \cont{4}(\cont{2}(\cont{2}(\cont{0})))),
  \null,\\
  & \br &
  \null :: \null :: [\null, (\cons~\nil~0)] :: [\null, (\cons~\nil~0)] :: \nils
  \rangle \\

  (14) & \estep & \langle
  \nil,
  \null,
  \null :: \null :: [\null, (\cons~\nil~0)] :: [\null, (\cons~\nil~0)] :: \nils,\\
  & \br &
  \cont{4}(\cont{4}(\cont{2}(\cont{2}(\cont{0}))))
  \rangle \\

  (15) & \estep & \langle
  \cont{4}(\cont{4}(\cont{2}(\cont{2}(\cont{0})))),
  \null,\\
  & \br &
  \null :: \null :: [\null, (\cons~\nil~0)] :: [\null, (\cons~\nil~0)] :: \nils,
  \rangle \\

  (16) & \estep & \langle
  \cont{4}(\cont{2}(\cont{2}(\cont{0}))),
  (\null, \null),\\
  & \br &
  \null :: [\null, (\cons~\nil~0)] :: [\null, (\cons~\nil~0)] :: \nils
  \rangle \\

  (17) & \estep & \langle
  \cont{2}(\cont{2}(\cont{0})),
  (\null, (\null, \null)),\\
  & \br &
  [\null, (\cons~\nil~0)] :: [\null, (\cons~\nil~0)] :: \nils
  \rangle \\

  (18) & \estep & \langle
  (\cons~\nil~0),
  (\null, (\null, (\null, \null))),
  [\null, (\cons~\nil~0)] :: \nils,
  \cont{2}(\cont{0})
  \rangle \\

  (19) & \estep & \langle
  \nil,
  (\null, (\null, (\null, \null))),
  (\null, (\null, (\null, \null))) :: [\null, (\cons~\nil~0)] :: \nils,\\
  & \br &
  \cont{3}(0, \cont{2}(\cont{0}))
  \rangle \\

  (20) & \estep & \langle
  \cont{3}(0, \cont{2}(\cont{0})),
  \null,
  (\null, (\null, (\null, \null))) :: [\null, (\cons~\nil~0)] :: \nils
  \rangle \\

  % TODO: Roe - type derivations for the next 10 steps.

  (21) & \estep & \langle
  0,
  (\null, (\null, (\null, \null))),
  \null :: [\null, (\cons~\nil~0)] :: \nils,
  \cont{4}(\cont{2}(\cont{0}))
  \rangle \\

  (22) & \estep & \langle
  \cont{4}(\cont{2}(\cont{0})),
  (\null, (\null, \null)),
  \null :: [\null, (\cons~\nil~0)] :: \nils
  \rangle \\

  (23) & \estep & \langle
  \cont{2}(\cont{0}),
  (\null, (\null, (\null, \null))),
  [\null, (\cons~\nil~0)] :: \nils
  \rangle \\

  (24) & \estep & \langle
  (\cons~\nil~0),
  (\null, (\null, (\null, (\null, \null)))),
  \nils,
  \cont{0}
  \rangle \\

  (25) & \estep & \langle
  \nil,
  (\null, (\null, (\null, (\null, \null)))),
  (\null, (\null, (\null, (\null, \null)))) :: \nils,
  \cont{3}(0, \cont{0})
  \rangle \\

  (26) & \estep & \langle
  \cont{3}(0, \cont{0}),
  \null,
  (\null, (\null, (\null, (\null, \null)))) :: \nils
  \rangle \\

  (27) & \estep & \langle
  0,
  (\null, (\null, (\null, (\null, \null)))),
  \null :: \nils,
  \cont{4}(\cont{0})
  \rangle \\

  (28) & \estep & \langle
  \cont{4}(\cont{0}),
  (\null, (\null, (\null, \null))),
  \null :: \nils
  \rangle \\

  (29) & \estep & \langle
  \cont{0},
  (\null, (\null, (\null, (\null, \null)))),
  \nils
  \rangle \\

  (30) & \estep & (\null, (\null, (\null, (\null, \null))))

\end{array}
\]

\section{Type Preservation}

Now, we will make an informal argument for type preservation by
showing that types are preserved at several steps of the evaluation of
the previous expression. We have chosen the steps to derive such that
we will use as many of the rules as possible. This expression does not
use \car or \cdr anywhere in its evaluation, so these rules are not
covered.

\newcommand{\fulltype}{\ensuremath{\tau_\null \times \tau_\null \times
    \tau_\null \times \tau_\null \times \tau_\null}}


Let $\tau = \fulltype$.

For convenience, let $\rho^n = \rho \times \ldots \times \rho$. For
example, ${\tau_\null}^3 = \tau_\null \times \tau_\null \times
\tau_\null$. Note that $\tau = {\tau_\null}^5 = \tau_\null \times {\tau_\null}^4$.
    

\subsection*{Step 1}
\[
\infer
{
	\vdash
	((\lambda (\cons~\nil~0))~
	 ((\lambda (\cons~\nil~0))~
	  (\cons~\nil~(\cons~\nil~\nil)))))
	: \tau_\null^5
}
{
	\infer
	{
		\vdash 
		(\lambda (\cons~\nil~0)) 
		: \tau_\null^4~\rightarrow~\tau_\null^5
	}
	{
	 \infer
	 {
	 	\tau_\null^4
	 	\vdash
	 	(\cons~\nil~0)
	 	: \tau_\null^5
	 }
	 {
	 	\tau_\null^4
	 	\vdash
	 	\nil
	 	: \tau_\null
	 	\quad
	 	\tau_\null^4
	 	\vdash
	 	0
	 	: \tau_\null^4	 		 	
	 }	 
	}
	\quad
	\infer
	{
		\vdash
		((\lambda (\cons~\nil~0))~
	  	 (\cons~\nil~(\cons~\nil~\nil)))
	 	: \tau_\null^4
	}
	{
		\textrm{Subtree (a)}
		\quad
		\textrm{Subtree (b)}
	}
}
\]

Subtree (a)
\[
\infer
{
	\vdash
	(\lambda (\cons~\nil~0))
	: \tau_\null^3\rightarrow\tau_\null^4
}
{
	\infer
	{
		\tau_\null^3
	 	\vdash
	 	(\cons~\nil~0)
	 	: \tau_\null^4
	 }
	 {		 
	 	\tau_\null^3
	 	\vdash
	 	\nil
	 	: \tau_\null		 				 		 	 
	 	\quad	 	 
	 	\tau_\null^3
	 	\vdash
	 	0
	 	: \tau_\null^3	 	
	 }	
}
\]

Subtree (b)
\[
\infer
{
	\vdash
	(\cons~\nil~(\cons~\nil~\nil))
	: \tau_\null^3
}
{
	\vdash
	\nil
	: \tau_\null	
	\quad
	\infer
	{
		\vdash
		(cons~\nil~\nil)
		: \tau_\null^2
	}
	{	
		\vdash
		\nil
		:\tau_\null	
		\quad		
		\vdash
		\nil
		:\tau_\null		
	}
}
\]

\subsection*{Step 3}

\[
\inferrule*
{
  	\inferrule*	
	{
	 	\inferrule*	 	
	 	{
	 		\tau_\null^4
	 		\vdash
	 		\nil
	 		: \tau_\null
	 		\+
	 		\tau_\null^4
	 		\vdash
	 		0
	 		: \tau_\null^4	 		 	
	 	}
	 	{
	 		\tau_\null^4
	 		\vdash
	 		(\cons~\nil~0)
	 		: \tau_\null^5
	 	}	 
	}
  	{  	    
    	\vdash
    	\lambda (\cons~\nil~0)
    	: {\tau_\null}^4 \rightarrow {\tau_\null}^5      
  	}~~~\textrm{Subtree (a)}~~~\vdash \null
}
{
	\vdash
	\langle
  		\lambda (\cons~\nil~0), %t
		\null,					%v
  		\null :: \nils,			%s
  		\cont{1}((\lambda (\cons~\nil~0))~(\cons~\nil~(\cons~\nil~\nil)),	%k
        	\cont{0})
  	\rangle 		
  	: \tau_\null^5
}
\]

Subtree(a)
\[
\inferrule*
{
	\inferrule*
	{
		\textrm{Step1-Subtree (a)}
		\+
		\textrm{Step1-Subtree (b)}
	}
	{
		\vdash
		(\lambda (\cons~\nil~0))~(\cons~\nil~(\cons~\nil~\nil))
		: {\tau_\null}^4
	}~~~\nils \vdash \cont{0} : {\tau_\null}^5 \rightarrow {\tau_\null}^5 ~~~\vdash
	\null
}
{
	\null :: \nils
    \vdash
    \cont{1}((\lambda (\cons~\nil~0))~(\cons~\nil~(\cons~\nil~\nil)),
        	\cont{0})
    : ({\tau_\null}^4 \rightarrow {\tau_\null}^5) \rightarrow {\tau_\null}^5
}
\]

\subsection*{Step 11}

Let $s_1 = [\null, (\cons~\nil~0)] :: [\null, (\cons~\nil~0)] :: \nils$.

Let $s_2 = [\null, (\cons~\nil~0)] :: \nils$.



\[
\infer
{
  \vdash 
  \langle
  (\cons~\nil~\nil),
  \null,
  \null :: s_1,
  \cont{4}(\cont{2}(\cont{2}(\cont{0})))
  \rangle
  : \tau
}
{
  \cdot \vdash \null 
  \quad
  \infer{
    \cdot \vdash (\cons~\nil~\nil) : \tau_\null \times \tau_\null
  }
  {
    \cdot \vdash \nil : \tau_\null
    \quad
    \cdot \vdash \nil : \tau_\null
  }
  \quad
  \infer{
    \null :: s_1 \vdash 
    \cont{4}(\cont{2}(\cont{2}(\cont{0})))
    : \tau_\null \times \tau_\null \rightarrow \tau
  }
  {
    \textrm{Subtree (a)}
  }
}
\]

Subtree (a)
\[
\infer{
  \null :: s_1 \vdash 
  \cont{4}(\cont{2}(\cont{2}(\cont{0})))
  : \tau_\null \times \tau_\null \rightarrow \tau
}
{
  \cdot \vdash \null : \tau_\null
  \quad
  \infer{
    [\null, (\cons~\nil~0)] :: s_2 \vdash \cont{2}(\cont{2}(\cont{0}))
    : {\tau_\null}^3 \rightarrow \tau
  }
  {
    \infer{
      \vdash [\null, (\cons~\nil~0)] 
      : {\tau_\null}^3
      \rightarrow \tau_\null \times {\tau_\null}^3
    }
    {
      \dot \vdash \null
      \quad
      \infer{
        {\tau_\null}^3 \vdash 
        (\cons~\nil~0) 
        : \tau_\null \times {\tau_\null}^3
      }
      {
        {\tau_\null}^3 \vdash 
        \nil : \tau_\null
        \quad
        {\tau_\null}^3 \vdash 
        0 : {\tau_\null}^3
      }
    }
    \quad
    \infer{
      [\null, (\cons~\nil~0)] :: \nils \vdash
      \cont{2}(\cont{0}) 
      : {\tau_\null}^4 \rightarrow \tau
    }
    {
      \textrm{Subtree (b)}
    }
  }
}
\]

Subtree (b)

\[
\infer{
  [\null, (\cons~\nil~0)] :: \nils \vdash
  \cont{2}(\cont{0}) 
  : {\tau_\null}^4 \rightarrow \tau
}
{
  \infer{
    \vdash [\null, (\cons~\nil~0)] : {\tau_\null}^4 \rightarrow \tau
  }
  {
    \cdot \vdash \null
    \quad
    \infer{
      {\tau_\null}^4 \vdash (\cons~\nil~0) : \tau
    }
    {
      {\tau_\null}^4 \vdash \nil : \tau_\null
      \quad
      {\tau_\null}^4 \vdash 0 : {\tau_\null}^4
    }
  }
  \quad
  \nils \vdash \cont{0} : \tau \rightarrow \tau
}
\]

\subsection*{Step 26}

\[
\inferrule*{
  \inferrule*{   
    {\tau_\null}^4 \vdash (\null, (\null, (\null, (\null, \null))))
    \+
    {\tau_\null}^4 \vdash
    0 : {\tau_\null}^4
    \+
    \nils \vdash \cont{0} 
    : \tau_\null \times {\tau_\null}^4 \rightarrow {\tau_\null}^5
  }
  {
    (\null, (\null, (\null, (\null, \null)))) :: \nils
    \vdash
    \cont{3}(0, \cont{0})
    : {\tau_\null} \rightarrow {\tau_\null}^5
  }~~~\vdash \null : {\tau_\null}
}
{
  \vdash 
  \langle
  \cont{3}(0, \cont{0}), % k
  \null,                 % v
  (\null, (\null, (\null, (\null, \null)))) :: \nils % s
  \rangle
  : {\tau_\null}^5
}
\]

\end{document}
